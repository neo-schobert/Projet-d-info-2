\documentclass[a4paper, 11pt, hidelinks]{article}
\usepackage{bookmark}
\usepackage[utf8]{inputenc} 
\usepackage[T1]{fontenc}
\usepackage{lmodern}
\usepackage{graphicx}
\usepackage[french]{babel}
\usepackage{geometry}
\usepackage{eucal}
\usepackage{caption}
\usepackage{float}
\usepackage{url}
\usepackage{amsmath}
\usepackage{amssymb}
\usepackage{color}
\usepackage{hyperref}
\usepackage{cancel}



\geometry{hmargin=2cm,vmargin=1.5cm}
\newcommand{\prp}{\large \textbf{Proposition :} \large}

\newcommand{\tm}{\large \textbf{Théoreme :} \large}

\newcommand{\ex}{\textcolor{green}{Exemple :} }

\newcommand{\dm}{\textcolor{red}{\textbf{Démo :} } }

\newcommand{\de}{\large \textbf{Définition} \large }

\newcommand{\rmq}{\textbf{Remarque :} }

\newcommand{\bs}{\bigskip}

\newcommand{\voca}{\textcolor{blue}{\textbf{Vocabulaire} } }

\newcommand{\trinom}[3]{\begin{pmatrix}
    #1 \\
    #2 \\
    #3
\end{pmatrix}}

\newcommand{\quadrinom}[4]{\begin{pmatrix}
    #1 \\
    #2 \\
    #3 \\
    #4 \\
\end{pmatrix}}

\newcommand{\pentanom}[5]{\begin{pmatrix}
    #1 \\
    #2 \\
    #3 \\
    #4 \\
    #5
\end{pmatrix}}

\newcommand{\hexanom}[6]{\begin{pmatrix}
    #1 \\
    #2 \\
    #3 \\
    #4 \\
    #5 \\
    #6 
\end{pmatrix}}

\newcommand{\serie}[2]{\displaystyle\sum_{#1 =0}^{+\infty} #2_{#1} }

\newcommand{\tend}{\underset{n \to + \infty}{\longrightarrow} }

\newcommand{\Lra}{\Leftrightarrow}

\newcommand{\lra}{\leftrightarrow}

\newcommand{\Ra}{\Rightarrow}

\newcommand{\ra}{\rightarrow}

\newcommand{\la}{\leftarrow}

\newcommand{\La}{\Leftarrow}

\newcommand{\dsum}[2]{\displaystyle\sum_{#1}^{#2} }

\newcommand{\dint}[2]{\displaystyle\int_{#1}^{#2} }

\newcommand{\ntend}{\underset{n \to + \infty}{\not \longrightarrow} }

\newenvironment{lmatrix}{$ \left|\begin{array}{l} }{\end{array}\right.$}

\begin{document}




\title{Projet d'informatique 2}
\author{Bregeon Pierre / Schobert Néo}

\maketitle

\tableofcontents


\newpage 


\section{Introduction}
Le projet se concentre sur la recherche des racines d'un polynôme unitaire. On s'interesse ici à la méthode de Durand-Kerner.

\bs 
Cette méthode commence par limiter le rayon de recherche de ces racines en constatant que si $\lambda$ est une racine de $P$ ($P$ est le polynôme concerné),
alors $|\lambda|\leq 1+ \displaystyle\max_{0\leq i \leq p-1} |a_i|$.

\bs
On se concentre alors pour la suite du projet sur les complexes répondant à la condition citée ci-dessus.


Pour la suite, on applique simplement l'algorithme de Durand-Kerner telle que décrite dans l'énoncé.

\bs 
\begin{enumerate}
    \item On initialise $deg(P)$ nombres complexes distincts qui ne sont pas racine de notre polynôme
    \item On construit alors $deg(P)$ suites par récurrence comme définies dans l'énoncé.
    \item On distingue deux cas :
    \begin{itemize}
        \item Si la méthode de Durand-Kerner converge, on fait un calcul de limite. Les limites de ces suites tendent chacune vers une racine de P distincte. (Non prouvé mais admis ici)
        \item Sinon, on ne peut rien en déduire. (On sait tout de même qu'il existe au moins une racine si $deg(P)\geq 1$ (Théorème d'Alembert))
    \end{itemize}
\end{enumerate}
\bs

Il nous faut aussi représenter notre polynôme $P$ de la manière la plus adaptée possible. Nous avons donc choisi la structure
du tableau. Ce tableau contiendra tous les coefficients du polynôme $P$ dans l'ordre des degrés décroissants 
\textbf{exprimés en notation complexe (exemple : $1 = 1 + 0j$)}\footnote{On les mettra en notation complexe pour éviter les problèmes 
liés aux conversions complexe / float} (sauf le premier qui vaut $1$ car $P$ est unitaire).
La structure du tableau a été choisi car il s'agit d'une structure statique et avec une indexation bien définie et faite pour.
Les listes auraient pu convenir mais elles n'ont pas été retenues car même si en python, elles ont une  indexation adaptée,
leur aspect dynamique nous est inutile. 


\section{Limitation du rayon de recherche}
Comme dit antérieurement\footnote{cf Introduction}, si $\lambda$ est une racine de $P$ ($P$ est le polynôme concerné),
alors : $$|\lambda|\leq 1+ \displaystyle\max_{0\leq i \leq p-1} |a_i|$$. (avec $\{ a_i,i \in [\![1,p-1]\!] \}$ l'ensemble des coefficients de $P$)
\bs

\subsection{Algorithme}
Le but ici est de limiter notre recherche. 
\bs

Sachant que l'on recherche les racines d'un polynôme et qu'elles se situent
toutes dans le disque $D=\{ z\in \mathbb{C}/ |z|\leq 1+ \displaystyle\max_{0\leq i \leq p-1} |a_i| \}$, nous allons déterminer 
la valeur de $1+\displaystyle\max_{0\leq i \leq p-1} |a_i|$.

La fonction calcul\_R(tab) du fichier .py permet alors de faire cela.

\subsection{Complexité}
La fonction calcul\_R fait :
\begin{itemize}
    \item deux affectations
    \item une boucle de longueur deg(P)
    \item une condition et une affectation
    \item une opération dans le return
\end{itemize}
\bs

La complexité est alors au pire (en comptant $1$ pour une affectation un calcul ou un test) :

$$c=2+deg(P)*2+1=3+2*deg(P)=\mathcal{O}(deg(P))$$


\section{Algorithme de Durand-Kerner}
Après un court raffinage, nous avons constaté que l'algorithme de Durand-Kerner doit être séparé en 3 fonctions :
\bs

\begin{itemize}
    \item Un première qui permettra de calculer $P(x_n^{k})$ appelée ici P\_el dans le fichier .py
    \item Une seconde qui permettra de générer un complexe aléatoire (fonction facultative)
    \item La troisième qui fera le reste de l'algorithme
\end{itemize}


\subsection{Algorithme}

Notre fonction prend deux arguments : le premier est le tableau représentant le polynôme, le second 
est un argument de précision, qui permet à l'utilisateur de choisir le degré de précision des racines 
en fonction de la complexité. On rappelle de plus que la complexité de la fonction croit avec le degré
du polynôme entré. (Il ne faut donc pas choisir un epsilon trop important)


\subsection{Complexité}
On comptera $1$ pour chaque affectation ou opération.

La première fonction (P\_el) fait :
\begin{itemize}
    \item deux affectation
    \item une boucle de longueur $deg(P)$ dans laquelle deux opérations et une affectation est faite
\end{itemize}.

La complexité associée est alors :

$$c=2+deg(P)*3=\mathcal{O}(deg(P))$$

La seconde fonction (random\_complex\_de\_D)) fait seulement $3$ opérations ($\mathcal{O}(1)$)

La troisième fonction (suite_Durand_Kerner1) fait :
\begin{itemize}
    \item trois affectations et une copie de tableau de taille $deg(P)$\footnote{la copie de tableau est en $\mathcal{O}(deg(P))$}
    \item boucle de longueur $deg(P)$ dans laquelle on a un while à condition quasiment toujours fausse (la terminaison n'est pas 
    assurée mais la probabilité fait que l'on peut considérer sa complexité en $\mathcal{O}(1)$)
    \item Une deuxième boucle (qui fera le coeur de notre complexité) dans laquelle on a :
    
\end{itemize}

\section{Utilisation et visualisation de la méthode de Durand-Kerner}





\section{Dépassements}



\section{Conclusion}







\end{document}